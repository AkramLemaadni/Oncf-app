\chapter{Contexte du projet et présentation de l'ONCF}

\section{Introduction}

Dans le cadre de notre Projet de Fin d'Année (PFA), nous avons développé une application web moderne pour l'Office National des Chemins de Fer (ONCF) du Maroc. Cette application vise à améliorer la gestion des ressources humaines et l'analyse des données opérationnelles au sein de l'entreprise.

Ce chapitre présente le contexte général du projet, l'organisme d'accueil ONCF, ainsi que les besoins identifiés qui ont motivé le développement de cette solution informatique.

\section{Présentation de l'ONCF}

\subsection{Historique et mission}

L'Office National des Chemins de Fer (ONCF) est l'opérateur ferroviaire national du Maroc, créé en 1963 par la fusion de plusieurs compagnies ferroviaires privées héritées de l'époque coloniale. L'ONCF est un établissement public à caractère commercial et industriel placé sous la tutelle du Ministère de l'Équipement, du Transport et de la Logistique.

L'entreprise a pour mission principale :
\begin{itemize}
    \item L'exploitation du réseau ferroviaire national
    \item Le transport de voyageurs et de marchandises
    \item Le développement et la maintenance de l'infrastructure ferroviaire
    \item La modernisation du système de transport ferroviaire marocain
\end{itemize}

\subsection{Chiffres clés}

L'ONCF dispose d'infrastructures importantes qui en font un acteur majeur du transport au Maroc :
\begin{itemize}
    \item \textbf{Réseau} : 3 600 km de voies ferrées en écartement standard
    \item \textbf{Électrification} : 1 300 km de lignes électrifiées à 3 000V
    \item \textbf{Personnel} : Environ 8 000 employés
    \item \textbf{Trafic} : Plus de 38 millions de voyageurs transportés annuellement
    \item \textbf{Fret} : 36 000 tonnes de marchandises transportées par an
\end{itemize}

\subsection{Modernisation et innovation}

L'ONCF s'est distingué par des projets innovants, notamment :
\begin{itemize}
    \item \textbf{LGV Al-Boraq} : Première ligne à grande vitesse d'Afrique (2018)
    \item \textbf{Digitalisation} : Modernisation des systèmes d'information
    \item \textbf{Extension du réseau} : Projets d'expansion vers le sud du pays
    \item \textbf{Amélioration de la qualité} : Programmes de formation et d'évaluation
\end{itemize}

\section{Problématique et besoins identifiés}

\subsection{Défis de gestion des ressources humaines}

L'ONCF, avec ses 8 000 employés répartis sur l'ensemble du territoire national, fait face à des défis importants en matière de gestion des ressources humaines :

\begin{itemize}
    \item \textbf{Évaluation du personnel} : Nécessité d'un système standardisé pour évaluer les performances des techniciens et ingénieurs
    \item \textbf{Suivi hiérarchique} : Besoin d'outils pour gérer les relations de supervision entre ingénieurs et techniciens
    \item \textbf{Traçabilité} : Manque de visibilité sur l'historique des évaluations et des performances
\end{itemize}

\subsection{Besoins en analyse de données}

L'exploitation d'un réseau ferroviaire génère une quantité importante de données opérationnelles qui nécessitent des outils d'analyse appropriés :

\begin{itemize}
    \item \textbf{Données de confort} : Analyse des évaluations de confort des trains
    \item \textbf{Incidents de transport} : Suivi et analyse des incidents pour améliorer la sécurité
    \item \textbf{Visualisation} : Besoin d'outils graphiques pour faciliter l'interprétation des données
    \item \textbf{Reporting} : Génération de rapports automatisés pour la direction
\end{itemize}

\subsection{Modernisation technologique}

L'ONCF s'inscrit dans une démarche de transformation numérique qui nécessite :

\begin{itemize}
    \item \textbf{Applications web modernes} : Remplacement des outils obsolètes
    \item \textbf{Interfaces utilisateur intuitives} : Adaptation aux besoins des différents profils d'utilisateurs
    \item \textbf{Accessibilité mobile} : Support des appareils mobiles pour les agents de terrain
    \item \textbf{Sécurité} : Mise en place de systèmes d'authentification robustes
\end{itemize}

\section{Solution proposée}

\subsection{Vue d'ensemble de l'application}

Pour répondre à ces besoins, nous avons développé une application web complète qui intègre :

\begin{itemize}
    \item \textbf{Gestion des utilisateurs} : Système d'authentification avec rôles différenciés (Ingénieur, Technicien, Stagiaire)
    \item \textbf{Système d'évaluation} : Interface pour que les ingénieurs puissent évaluer leurs techniciens
    \item \textbf{Analyse de données} : Module de traitement et visualisation des données de trains
    \item \textbf{Tableaux de bord} : Interfaces personnalisées selon le profil utilisateur
    \item \textbf{Support multilingue} : Application disponible en français et anglais
\end{itemize}

\subsection{Fonctionnalités principales}

\subsubsection{Module de gestion des utilisateurs}
\begin{itemize}
    \item Inscription et authentification sécurisées
    \item Gestion des rôles et permissions
    \item Hiérarchie : association techniciens-ingénieurs superviseurs
\end{itemize}

\subsubsection{Système d'évaluation}
\begin{itemize}
    \item Formulaires d'évaluation avec notation (1-10) et commentaires
    \item Contrôle d'accès : seuls les superviseurs peuvent évaluer leurs techniciens
    \item Historique des évaluations avec traçabilité complète
\end{itemize}

\subsubsection{Analyse de données de trains}
\begin{itemize}
    \item Upload de fichiers Excel contenant les données opérationnelles
    \item Génération automatique de graphiques (courbes, barres, nuages de points)
    \item Analyse spécialisée des incidents de transport
    \item Export et sauvegarde des analyses
\end{itemize}

\subsection{Architecture technique}

L'application est construite avec des technologies modernes :

\begin{itemize}
    \item \textbf{Backend} : Spring Boot 3.5, Spring Security, JPA/Hibernate
    \item \textbf{Frontend} : Thymeleaf, Bootstrap 5, Chart.js pour les visualisations
    \item \textbf{Base de données} : MySQL 8.0
    \item \textbf{Sécurité} : JWT (JSON Web Tokens) pour l'authentification
    \item \textbf{Gestion des fichiers} : Support d'upload et traitement des fichiers Excel
\end{itemize}

\section{Objectifs du projet}

\subsection{Objectifs fonctionnels}

\begin{itemize}
    \item Développer un système d'évaluation digitalisé pour améliorer le suivi des performances
    \item Créer des outils d'analyse de données accessibles aux ingénieurs
    \item Implémenter une solution multilingue adaptée au contexte marocain
    \item Assurer la traçabilité complète des évaluations et analyses
\end{itemize}

\subsection{Objectifs techniques}

\begin{itemize}
    \item Maîtriser le framework Spring Boot et l'écosystème Spring
    \item Implémenter une sécurité robuste avec Spring Security
    \item Développer des interfaces utilisateur modernes et responsives
    \item Intégrer des fonctionnalités de visualisation de données
\end{itemize}

\subsection{Objectifs pédagogiques}

\begin{itemize}
    \item Appliquer les connaissances théoriques en développement web
    \item Découvrir les enjeux du développement en entreprise
    \item Comprendre les besoins métier d'une grande organisation publique
    \item Développer des compétences en gestion de projet informatique
\end{itemize}

\section{Conclusion}

L'ONCF, en tant qu'acteur majeur du transport ferroviaire marocain, nécessite des outils informatiques modernes pour accompagner sa transformation numérique. Notre application répond à des besoins concrets en matière de gestion des ressources humaines et d'analyse de données opérationnelles.

Ce projet s'inscrit dans une démarche d'amélioration continue des processus internes de l'ONCF et constitue une base solide pour de futurs développements. Les chapitres suivants détailleront les aspects techniques de conception et de réalisation de cette solution.