\chapter{Présentation de l'entreprise ONCF}

\section{Introduction}

L'Office National des Chemins de Fer (ONCF) constitue l'épine dorsale du transport ferroviaire au Maroc. Créé en 1963, cet établissement public à caractère commercial et industriel s'est imposé comme un acteur majeur du développement économique et social du Royaume. Dans le cadre de notre projet de fin d'études, il est essentiel de comprendre l'environnement dans lequel s'inscrit notre travail.

Ce chapitre présente une vue d'ensemble de l'ONCF, de son histoire, de sa mission, de son organisation, ainsi que de sa stratégie de développement. Cette présentation permettra de mieux appréhender les enjeux et les défis auxquels fait face l'entreprise, et par conséquent, de situer notre contribution dans son contexte approprié.

\section{Historique et création de l'ONCF}

\subsection{Les origines du transport ferroviaire au Maroc}

Le transport ferroviaire au Maroc trouve ses origines durant la période du protectorat français (1912-1956). Les premières lignes ferroviaires ont été construites par différentes compagnies privées pour répondre aux besoins économiques et stratégiques de l'époque coloniale.

Les principales compagnies ferroviaires qui ont précédé l'ONCF étaient :
\begin{itemize}
    \item \textbf{CFM (Compagnie des Chemins de Fer du Maroc)} : La plus grande compagnie du groupe, responsable du réseau principal
    \item \textbf{CMO (Compagnie du Chemin de Fer du Maroc Oriental)} : Filiale de CFM gérant la partie orientale du réseau
    \item \textbf{TF (Compagnie franco-espagnole du Tanger-Fès)} : Responsable de la ligne Tanger-Fès et des branches associées
    \item \textbf{MN (Chemins de Fer de la Méditerranée au Niger)} : Dont l'ONCF a hérité la ligne vers Bouarfa dans le désert du Sahara
\end{itemize}

\subsection{La création de l'ONCF en 1963}

Après l'indépendance du Maroc en 1956, l'État marocain a entrepris une politique de nationalisation des infrastructures stratégiques. C'est dans ce contexte que l'ONCF a été créé en 1963 par décision gouvernementale, résultant de la fusion des différentes compagnies ferroviaires privées héritées de l'époque coloniale.

Cette création s'inscrivait dans une vision globale de développement national, visant à :
\begin{itemize}
    \item Unifier le réseau ferroviaire sous une gestion nationale cohérente
    \item Moderniser et étendre l'infrastructure ferroviaire
    \item Développer le transport de voyageurs et de marchandises
    \item Contribuer au développement économique et social du pays
\end{itemize}

\section{Statut juridique et mission}

\subsection{Statut juridique}

L'ONCF est un établissement public à caractère commercial et industriel doté de l'autonomie financière. Il est placé sous la tutelle du Ministère de l'Équipement, du Transport et de la Logistique. Ce statut lui confère une flexibilité de gestion tout en maintenant sa mission de service public.

\subsection{Mission et objectifs}

L'ONCF a pour mission principale :
\begin{itemize}
    \item \textbf{Exploitation du réseau ferroviaire} : Assurer le transport de voyageurs et de marchandises sur l'ensemble du territoire national
    \item \textbf{Développement de l'infrastructure} : Étudier, construire et maintenir les lignes ferroviaires
    \item \textbf{Services connexes} : Exploiter tous les services liés au transport ferroviaire, tant au niveau local que national
\end{itemize}

\section{Organisation et structure}

\subsection{Direction générale}

L'ONCF est dirigé par un Directeur Général, actuellement Monsieur Mohamed Rabie Khlie. La direction générale a son siège à Rabat, précisément au 8, rue Abderrahmane El Ghafiki, Agdal.

\subsection{Ressources humaines}

L'ONCF emploie environ 7 845 collaborateurs répartis sur l'ensemble du territoire national. Ces effectifs comprennent différentes catégories professionnelles :
\begin{itemize}
    \item Cadres techniques et ingénieurs
    \item Agents d'exploitation ferroviaire
    \item Personnel de maintenance
    \item Personnel administratif et commercial
    \item Agents de sécurité et de sûreté
\end{itemize}

\section{Infrastructure et réseau}

\subsection{Caractéristiques du réseau}

Le réseau ferroviaire marocain géré par l'ONCF s'étend sur 3 600 kilomètres, entièrement en écartement standard (1 435 mm). Parmi cette infrastructure :
\begin{itemize}
    \item \textbf{1 300 km électrifiés} à 3 000 volts courant continu
    \item \textbf{Lignes principales} : Nord-Sud (Tanger-Marrakech) et Est-Ouest (Oujda-Rabat)
    \item \textbf{Lignes secondaires} reliant les villes côtières et l'arrière-pays
\end{itemize}

\subsection{Modernisation et projets structurants}

L'ONCF a engagé d'importants programmes de modernisation :

\subsubsection{Ligne à Grande Vitesse Al-Boraq}
Inaugurée en 2018, la LGV Al-Boraq représente une première en Afrique et dans le monde arabe. Cette ligne de 186 km relie Tanger à Casablanca en 2h10, avec une vitesse commerciale de 320 km/h. Le projet, développé avec l'assistance technique française (Alstom), constitue un symbole de modernité et de performance.

\subsubsection{Autres projets d'extension}
\begin{itemize}
    \item Extension Casablanca-El Jadida
    \item Projet Marrakech-Agadir
    \item Contournement de Meknès sur la ligne Rabat-Fès
\end{itemize}

\section{Activités et services}

\subsection{Transport de voyageurs}

L'ONCF transporte annuellement plus de 38 millions de voyageurs (données 2013) à travers différents types de services :
\begin{itemize}
    \item \textbf{Al-Boraq} : Service à grande vitesse
    \item \textbf{Trains InterCités} : Services longue distance avec confort moderne
    \item \textbf{TNR (Train Navette Rapide)} : Services régionaux à haute fréquence
    \item \textbf{Trains de nuit} : Services couchettes sur les principales liaisons
    \item \textbf{Supratours} : Services d'autocars de rabattement
\end{itemize}

\subsection{Transport de marchandises}

Le fret ferroviaire représente un secteur stratégique avec 36 200 tonnes transportées annuellement. Les principales activités incluent :
\begin{itemize}
    \item Transport de phosphates (secteur majeur)
    \item Transport de marchandises diverses
    \item Logistique industrielle
    \item Transport conteneurisé
\end{itemize}

\section{Stratégie et vision d'avenir}

\subsection{Plan stratégique}

L'ONCF s'inscrit dans une démarche de développement durable et de modernisation continue. Ses orientations stratégiques portent sur :
\begin{itemize}
    \item Développement du transport à grande vitesse
    \item Amélioration de la qualité de service
    \item Extension du réseau vers le sud (projet Marrakech-Agadir)
    \item Modernisation du matériel roulant
    \item Développement du fret ferroviaire
\end{itemize}

\subsection{Enjeux et défis}

L'ONCF fait face à plusieurs défis majeurs :
\begin{itemize}
    \item Concurrence du transport routier
    \item Financement des investissements lourds
    \item Adaptation aux nouvelles technologies
    \item Satisfaction croissante des attentes clients
    \item Contribution aux objectifs environnementaux nationaux
\end{itemize}

\section{Conclusion}

L'ONCF représente un acteur clé du développement économique marocain. Avec plus de 60 ans d'expérience, l'entreprise a su évoluer et se moderniser pour répondre aux défis contemporains du transport ferroviaire. Son engagement dans l'innovation, illustré par la réalisation de la première ligne à grande vitesse africaine, témoigne de sa volonté de maintenir sa position de leader régional.

Dans le contexte de notre projet de fin d'études, cette présentation de l'ONCF nous permet de mieux comprendre l'environnement dans lequel s'inscrit notre travail et les enjeux auxquels l'entreprise fait face en matière de modernisation de ses systèmes d'information et de gestion.