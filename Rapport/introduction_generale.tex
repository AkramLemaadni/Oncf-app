\chapter*{Introduction générale}
\addcontentsline{toc}{chapter}{Introduction générale}

Dans un contexte de transformation numérique accélérée et de modernisation des services publics, les entreprises marocaines font face à des défis considérables pour améliorer leur efficacité opérationnelle et la qualité de leurs services. L'Office National des Chemins de Fer (ONCF), en tant qu'acteur majeur du transport ferroviaire au Maroc, n'échappe pas à cette dynamique de modernisation.

Créé en 1963, l'ONCF gère aujourd'hui un réseau ferroviaire de 3 600 kilomètres et emploie près de 8 000 collaborateurs. L'entreprise assure le transport de plus de 38 millions de voyageurs par an et joue un rôle crucial dans le développement économique du Royaume. Avec des projets ambitieux comme la ligne à grande vitesse Al-Boraq, première du genre en Afrique, l'ONCF démontre sa volonté d'innovation et de modernisation.

Cependant, cette croissance et cette modernisation s'accompagnent de nouveaux défis organisationnels et opérationnels. La gestion des évaluations de confort des trains, le suivi des performances, l'administration des ressources humaines et la coordination entre les différents services nécessitent des outils informatiques adaptés et performants.

Dans ce contexte, notre projet de fin d'études s'inscrit dans une démarche d'amélioration des processus internes de l'ONCF par le développement d'une application web moderne et intuitive. Cette application vise à digitaliser et optimiser la gestion des évaluations de confort des trains, tout en offrant des outils de pilotage et de suivi statistique aux différents acteurs de l'entreprise.

\section*{Problématique}

L'ONCF fait face à plusieurs défis dans la gestion de ses opérations quotidiennes :
\begin{itemize}
    \item La nécessité de centraliser et standardiser les processus d'évaluation de confort des trains
    \item Le besoin d'outils de suivi et d'analyse des performances opérationnelles
    \item L'importance de disposer d'interfaces utilisateur adaptées aux différents profils métier
    \item La volonté d'améliorer la traçabilité et l'historique des évaluations
\end{itemize}

Face à ces enjeux, comment développer une solution informatique qui réponde aux besoins spécifiques de l'ONCF tout en s'intégrant harmonieusement dans son écosystème technologique existant ?

\section*{Objectifs}

Ce projet vise à atteindre plusieurs objectifs :

\textbf{Objectif principal :} Développer une application web moderne pour la gestion des évaluations de confort des trains à l'ONCF.

\textbf{Objectifs spécifiques :}
\begin{itemize}
    \item Concevoir une architecture logicielle robuste et évolutive
    \item Implémenter un système d'authentification et de gestion des rôles utilisateurs
    \item Créer des interfaces utilisateur intuitives et responsives
    \item Développer un module de saisie et de gestion des évaluations de confort
    \item Intégrer des fonctionnalités de visualisation et d'analyse statistique
    \item Assurer la sécurité et la protection des données
    \item Garantir la compatibilité multilingue (français/arabe)
\end{itemize}

\section*{Approche méthodologique}

Notre approche s'articule autour de plusieurs axes :
\begin{itemize}
    \item \textbf{Analyse des besoins :} Étude approfondie du contexte ONCF et identification des exigences fonctionnelles et techniques
    \item \textbf{Conception :} Modélisation de l'architecture logicielle et des interfaces utilisateur
    \item \textbf{Développement :} Implémentation en utilisant les technologies Spring Boot et les standards web modernes
    \item \textbf{Tests et validation :} Vérification de la conformité aux exigences et optimisation des performances
\end{itemize}

\section*{Structure du rapport}

Ce rapport s'organise en plusieurs chapitres :

Le \textbf{premier chapitre} présente l'entreprise ONCF, son historique, sa mission, son organisation et ses enjeux stratégiques.

Le \textbf{deuxième chapitre} expose l'étude bibliographique et l'état de l'art des technologies utilisées dans le développement d'applications web modernes.

Le \textbf{troisième chapitre} détaille l'analyse et la conception de notre solution, incluant l'étude des besoins, la modélisation et l'architecture proposée.

Le \textbf{quatrième chapitre} présente la réalisation technique du projet, les choix d'implémentation et les fonctionnalités développées.

Enfin, la \textbf{conclusion générale} synthétise les résultats obtenus, évalue l'atteinte des objectifs fixés et propose des perspectives d'amélioration et d'évolution du système.

\vfill

\newpage