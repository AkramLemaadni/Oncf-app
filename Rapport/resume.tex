\chapter*{Résumé}
\addcontentsline{toc}{chapter}{Résumé}

\begin{justify}

Ce projet de fin d'année (PFA) a été réalisé au sein de l'Office National des Chemins de Fer (ONCF), établissement public marocain créé en 1963, gérant un réseau ferroviaire de 3 600 kilomètres et employant près de 8 000 collaborateurs. Face aux défis de modernisation et d'optimisation de ses processus métier, l'ONCF cherche à digitaliser ses opérations pour améliorer son efficacité et la qualité de ses services. Le projet consiste en la conception et le développement d'une application web moderne destinée à améliorer la gestion des ressources humaines et l'analyse des données opérationnelles au sein de l'entreprise. L'application développée utilise une architecture client-serveur moderne avec Spring Boot pour le backend, Thymeleaf pour le frontend, et MySQL pour la base de données. Les fonctionnalités principales incluent un système d'authentification sécurisé avec JWT, une gestion complète des utilisateurs (Ingénieurs, Techniciens, Stagiaires), un système d'évaluation des techniciens par les ingénieurs, et un module d'analyse statistique permettant l'upload et l'analyse de données de trains via des fichiers Excel. L'interface utilisateur est responsive et multilingue (français/anglais), intégrant des graphiques dynamiques avec Chart.js pour la visualisation des données. Les exigences non-fonctionnelles privilégient la performance, la sécurité (HTTPS, RBAC, JWT), la maintenabilité et l'évolutivité. Le développement a suivi une méthodologie agile, utilisant des outils modernes comme IntelliJ IDEA et Git. L'implémentation fournit une solution complète pour les besoins opérationnels de l'ONCF, contribuant à sa stratégie de transformation numérique et démontrant l'importance des systèmes d'information dans la modernisation des entreprises publiques marocaines.

\end{justify}

\textbf{Mots-clés :} ONCF, Application web, Spring Boot, Gestion RH, Analyse de données, Système d'information, Digitalisation

\vfill

\newpage