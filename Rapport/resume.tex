\chapter*{Résumé}
\addcontentsline{toc}{chapter}{Résumé}

Dans le cadre de notre projet de fin d'études réalisé au sein de l'Office National des Chemins de Fer (ONCF), nous avons développé une application web moderne destinée à améliorer la gestion des activités opérationnelles de l'entreprise.

L'ONCF, établissement public marocain créé en 1963, gère un réseau ferroviaire de 3 600 kilomètres et emploie près de 8 000 collaborateurs. Face aux défis de modernisation et d'optimisation de ses processus métier, l'entreprise cherche à digitaliser ses opérations pour améliorer son efficacité et la qualité de ses services.

Notre projet consiste en la conception et le développement d'une application web utilisant les technologies modernes Spring Boot pour le backend et des technologies web standards (HTML5, CSS3, JavaScript) pour le frontend. Cette application permet la gestion des évaluations de confort des trains, l'administration des utilisateurs, et le suivi statistique des performances.

L'application développée offre plusieurs fonctionnalités clés : un système d'authentification sécurisé, une interface de saisie d'évaluations de confort pour les techniciens, un tableau de bord statistique pour les ingénieurs, et une gestion complète des utilisateurs. L'interface utilisateur est responsive et multilingue (français/arabe), s'adaptant aux besoins spécifiques du contexte marocain.

Les technologies utilisées incluent Spring Boot pour le framework backend, Spring Security pour la sécurité, JPA/Hibernate pour la persistance des données, et une base de données relationnelle pour le stockage. Le frontend utilise Bootstrap pour l'interface responsive et Chart.js pour la visualisation des données statistiques.

Ce projet contribue à la digitalisation des processus de l'ONCF et démontre l'importance des systèmes d'information dans la modernisation des entreprises publiques marocaines. Il s'inscrit dans la stratégie de transformation numérique de l'ONCF visant à améliorer l'efficacité opérationnelle et la qualité de service.

\textbf{Mots-clés :} ONCF, Application web, Spring Boot, Gestion des évaluations, Système d'information, Digitalisation, Transport ferroviaire

\vfill

\newpage